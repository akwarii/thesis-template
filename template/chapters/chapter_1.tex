\chapter{Ceci est un chapitre}

\epigraph{Bien entendu les épigraphes sont optionnelles}{Toujours moi, \textit{chez moi}}

\section{Première section}

\insertfigure[scale=.3]{logos.jpg}

\subsection{Sous-section}
Vous remarquerez que les sous-sous-sections ne sont pas numérotées par défaut. C'est un choix de style\index{style}. Vous pouvez les numéroter en commentant la ligne \texttt{\\setcounter\{secnumdepth\}\{2\}} dans le fichier \texttt{preamble.tex}.

\subsubsection{Sous-sous-section}
% \lipsum[1-8]
Vous remarquerez que les sous-sous-sections ne sont pas numérotées par défaut. C'est un choix de style\index{style}. Vous pouvez les numéroter en commentant la ligne \texttt{\\setcounter\{secnumdepth\}\{2\}} dans le fichier \texttt{preamble.tex}.
Vous remarquerez que les sous-sous-sections ne sont pas numérotées par défaut. C'est un choix de style\index{style}. Vous pouvez les numéroter en commentant la ligne \texttt{\\setcounter\{secnumdepth\}\{2\}} dans le fichier \texttt{preamble.tex}.
Vous remarquerez que les sous-sous-sections ne sont pas numérotées par défaut. C'est un choix de style\index{style}. Vous pouvez les numéroter en commentant la ligne \texttt{\\setcounter\{secnumdepth\}\{2\}} dans le fichier \texttt{preamble.tex}.
Vous remarquerez que les sous-sous-sections ne sont pas numérotées par défaut. C'est un choix de style\index{style}. Vous pouvez les numéroter en commentant la ligne \texttt{\\setcounter\{secnumdepth\}\{2\}} dans le fichier \texttt{preamble.tex}.
Vous remarquerez que les sous-sous-sections ne sont pas numérotées par défaut. C'est un choix de style\index{style}. Vous pouvez les numéroter en commentant la ligne \texttt{\\setcounter\{secnumdepth\}\{2\}} dans le fichier \texttt{preamble.tex}.
Vous remarquerez que les sous-sous-sections ne sont pas numérotées par défaut. C'est un choix de style\index{style}. Vous pouvez les numéroter en commentant la ligne \texttt{\\setcounter\{secnumdepth\}\{2\}} dans le fichier \texttt{preamble.tex}.
Vous remarquerez que les sous-sous-sections ne sont pas numérotées par défaut. C'est un choix de style\index{style}. Vous pouvez les numéroter en commentant la ligne \texttt{\\setcounter\{secnumdepth\}\{2\}} dans le fichier \texttt{preamble.tex}.
Vous remarquerez que les sous-sous-sections ne sont pas numérotées par défaut. C'est un choix de style\index{style}. Vous pouvez les numéroter en commentant la ligne \texttt{\\setcounter\{secnumdepth\}\{2\}} dans le fichier \texttt{preamble.tex}.
Vous remarquerez que les sous-sous-sections ne sont pas numérotées par défaut. C'est un choix de style\index{style}. Vous pouvez les numéroter en commentant la ligne \texttt{\\setcounter\{secnumdepth\}\{2\}} dans le fichier \texttt{preamble.tex}.
Vous remarquerez que les sous-sous-sections ne sont pas numérotées par défaut. C'est un choix de style\index{style}. Vous pouvez les numéroter en commentant la ligne \texttt{\\setcounter\{secnumdepth\}\{2\}} dans le fichier \texttt{preamble.tex}.
Vous remarquerez que les sous-sous-sections ne sont pas numérotées par défaut. C'est un choix de style\index{style}. Vous pouvez les numéroter en commentant la ligne \texttt{\\setcounter\{secnumdepth\}\{2\}} dans le fichier \texttt{preamble.tex}.
Vous remarquerez que les sous-sous-sections ne sont pas numérotées par défaut. C'est un choix de style\index{style}. Vous pouvez les numéroter en commentant la ligne \texttt{\\setcounter\{secnumdepth\}\{2\}} dans le fichier \texttt{preamble.tex}.
Vous remarquerez que les sous-sous-sections ne sont pas numérotées par défaut. C'est un choix de style\index{style}. Vous pouvez les numéroter en commentant la ligne \texttt{\\setcounter\{secnumdepth\}\{2\}} dans le fichier \texttt{preamble.tex}.
Vous remarquerez que les sous-sous-sections ne sont pas numérotées par défaut. C'est un choix de style\index{style}. Vous pouvez les numéroter en commentant la ligne \texttt{\\setcounter\{secnumdepth\}\{2\}} dans le fichier \texttt{preamble.tex}.
Vous remarquerez que les sous-sous-sections ne sont pas numérotées par défaut. C'est un choix de style\index{style}. Vous pouvez les numéroter en commentant la ligne \texttt{\\setcounter\{secnumdepth\}\{2\}} dans le fichier \texttt{preamble.tex}.
Vous remarquerez que les sous-sous-sections ne sont pas numérotées par défaut. C'est un choix de style\index{style}. Vous pouvez les numéroter en commentant la ligne \texttt{\\setcounter\{secnumdepth\}\{2\}} dans le fichier \texttt{preamble.tex}.
Vous remarquerez que les sous-sous-sections ne sont pas numérotées par défaut. C'est un choix de style\index{style}. Vous pouvez les numéroter en commentant la ligne \texttt{\\setcounter\{secnumdepth\}\{2\}} dans le fichier \texttt{preamble.tex}.
Vous remarquerez que les sous-sous-sections ne sont pas numérotées par défaut. C'est un choix de style\index{style}. Vous pouvez les numéroter en commentant la ligne \texttt{\\setcounter\{secnumdepth\}\{2\}} dans le fichier \texttt{preamble.tex}.
Vous remarquerez que les sous-sous-sections ne sont pas numérotées par défaut. C'est un choix de style\index{style}. Vous pouvez les numéroter en commentant la ligne \texttt{\\setcounter\{secnumdepth\}\{2\}} dans le fichier \texttt{preamble.tex}.
Vous remarquerez que les sous-sous-sections ne sont pas numérotées par défaut. C'est un choix de style\index{style}. Vous pouvez les numéroter en commentant la ligne \texttt{\\setcounter\{secnumdepth\}\{2\}} dans le fichier \texttt{preamble.tex}.
Vous remarquerez que les sous-sous-sections ne sont pas numérotées par défaut. C'est un choix de style\index{style}. Vous pouvez les numéroter en commentant la ligne \texttt{\\setcounter\{secnumdepth\}\{2\}} dans le fichier \texttt{preamble.tex}.
Vous remarquerez que les sous-sous-sections ne sont pas numérotées par défaut. C'est un choix de style\index{style}. Vous pouvez les numéroter en commentant la ligne \texttt{\\setcounter\{secnumdepth\}\{2\}} dans le fichier \texttt{preamble.tex}.
Vous remarquerez que les sous-sous-sections ne sont pas numérotées par défaut. C'est un choix de style\index{style}. Vous pouvez les numéroter en commentant la ligne \texttt{\\setcounter\{secnumdepth\}\{2\}} dans le fichier \texttt{preamble.tex}.
Vous remarquerez que les sous-sous-sections ne sont pas numérotées par défaut. C'est un choix de style\index{style}. Vous pouvez les numéroter en commentant la ligne \texttt{\\setcounter\{secnumdepth\}\{2\}} dans le fichier \texttt{preamble.tex}.
Vous remarquerez que les sous-sous-sections ne sont pas numérotées par défaut. C'est un choix de style\index{style}. Vous pouvez les numéroter en commentant la ligne \texttt{\\setcounter\{secnumdepth\}\{2\}} dans le fichier \texttt{preamble.tex}.
Vous remarquerez que les sous-sous-sections ne sont pas numérotées par défaut. C'est un choix de style\index{style}. Vous pouvez les numéroter en commentant la ligne \texttt{\\setcounter\{secnumdepth\}\{2\}} dans le fichier \texttt{preamble.tex}.
Vous remarquerez que les sous-sous-sections ne sont pas numérotées par défaut. C'est un choix de style\index{style}. Vous pouvez les numéroter en commentant la ligne \texttt{\\setcounter\{secnumdepth\}\{2\}} dans le fichier \texttt{preamble.tex}.
Vous remarquerez que les sous-sous-sections ne sont pas numérotées par défaut. C'est un choix de style\index{style}. Vous pouvez les numéroter en commentant la ligne \texttt{\\setcounter\{secnumdepth\}\{2\}} dans le fichier \texttt{preamble.tex}.
Vous remarquerez que les sous-sous-sections ne sont pas numérotées par défaut. C'est un choix de style\index{style}. Vous pouvez les numéroter en commentant la ligne \texttt{\\setcounter\{secnumdepth\}\{2\}} dans le fichier \texttt{preamble.tex}.
Vous remarquerez que les sous-sous-sections ne sont pas numérotées par défaut. C'est un choix de style\index{style}. Vous pouvez les numéroter en commentant la ligne \texttt{\\setcounter\{secnumdepth\}\{2\}} dans le fichier \texttt{preamble.tex}.
Vous remarquerez que les sous-sous-sections ne sont pas numérotées par défaut. C'est un choix de style\index{style}. Vous pouvez les numéroter en commentant la ligne \texttt{\\setcounter\{secnumdepth\}\{2\}} dans le fichier \texttt{preamble.tex}.
Vous remarquerez que les sous-sous-sections ne sont pas numérotées par défaut. C'est un choix de style\index{style}. Vous pouvez les numéroter en commentant la ligne \texttt{\\setcounter\{secnumdepth\}\{2\}} dans le fichier \texttt{preamble.tex}.
Vous remarquerez que les sous-sous-sections ne sont pas numérotées par défaut. C'est un choix de style\index{style}. Vous pouvez les numéroter en commentant la ligne \texttt{\\setcounter\{secnumdepth\}\{2\}} dans le fichier \texttt{preamble.tex}.
Vous remarquerez que les sous-sous-sections ne sont pas numérotées par défaut. C'est un choix de style\index{style}. Vous pouvez les numéroter en commentant la ligne \texttt{\\setcounter\{secnumdepth\}\{2\}} dans le fichier \texttt{preamble.tex}.
Vous remarquerez que les sous-sous-sections ne sont pas numérotées par défaut. C'est un choix de style\index{style}. Vous pouvez les numéroter en commentant la ligne \texttt{\\setcounter\{secnumdepth\}\{2\}} dans le fichier \texttt{preamble.tex}.
Vous remarquerez que les sous-sous-sections ne sont pas numérotées par défaut. C'est un choix de style\index{style}. Vous pouvez les numéroter en commentant la ligne \texttt{\\setcounter\{secnumdepth\}\{2\}} dans le fichier \texttt{preamble.tex}.
Vous remarquerez que les sous-sous-sections ne sont pas numérotées par défaut. C'est un choix de style\index{style}. Vous pouvez les numéroter en commentant la ligne \texttt{\\setcounter\{secnumdepth\}\{2\}} dans le fichier \texttt{preamble.tex}.
Vous remarquerez que les sous-sous-sections ne sont pas numérotées par défaut. C'est un choix de style\index{style}. Vous pouvez les numéroter en commentant la ligne \texttt{\\setcounter\{secnumdepth\}\{2\}} dans le fichier \texttt{preamble.tex}.
Vous remarquerez que les sous-sous-sections ne sont pas numérotées par défaut. C'est un choix de style\index{style}. Vous pouvez les numéroter en commentant la ligne \texttt{\\setcounter\{secnumdepth\}\{2\}} dans le fichier \texttt{preamble.tex}.
Vous remarquerez que les sous-sous-sections ne sont pas numérotées par défaut. C'est un choix de style\index{style}. Vous pouvez les numéroter en commentant la ligne \texttt{\\setcounter\{secnumdepth\}\{2\}} dans le fichier \texttt{preamble.tex}.
Vous remarquerez que les sous-sous-sections ne sont pas numérotées par défaut. C'est un choix de style\index{style}. Vous pouvez les numéroter en commentant la ligne \texttt{\\setcounter\{secnumdepth\}\{2\}} dans le fichier \texttt{preamble.tex}.
Vous remarquerez que les sous-sous-sections ne sont pas numérotées par défaut. C'est un choix de style\index{style}. Vous pouvez les numéroter en commentant la ligne \texttt{\\setcounter\{secnumdepth\}\{2\}} dans le fichier \texttt{preamble.tex}.
Vous remarquerez que les sous-sous-sections ne sont pas numérotées par défaut. C'est un choix de style\index{style}. Vous pouvez les numéroter en commentant la ligne \texttt{\\setcounter\{secnumdepth\}\{2\}} dans le fichier \texttt{preamble.tex}.
Vous remarquerez que les sous-sous-sections ne sont pas numérotées par défaut. C'est un choix de style\index{style}. Vous pouvez les numéroter en commentant la ligne \texttt{\\setcounter\{secnumdepth\}\{2\}} dans le fichier \texttt{preamble.tex}.
Vous remarquerez que les sous-sous-sections ne sont pas numérotées par défaut. C'est un choix de style\index{style}. Vous pouvez les numéroter en commentant la ligne \texttt{\\setcounter\{secnumdepth\}\{2\}} dans le fichier \texttt{preamble.tex}.
Vous remarquerez que les sous-sous-sections ne sont pas numérotées par défaut. C'est un choix de style\index{style}. Vous pouvez les numéroter en commentant la ligne \texttt{\\setcounter\{secnumdepth\}\{2\}} dans le fichier \texttt{preamble.tex}.
Vous remarquerez que les sous-sous-sections ne sont pas numérotées par défaut. C'est un choix de style\index{style}. Vous pouvez les numéroter en commentant la ligne \texttt{\\setcounter\{secnumdepth\}\{2\}} dans le fichier \texttt{preamble.tex}.
Vous remarquerez que les sous-sous-sections ne sont pas numérotées par défaut. C'est un choix de style\index{style}. Vous pouvez les numéroter en commentant la ligne \texttt{\\setcounter\{secnumdepth\}\{2\}} dans le fichier \texttt{preamble.tex}.
Vous remarquerez que les sous-sous-sections ne sont pas numérotées par défaut. C'est un choix de style\index{style}. Vous pouvez les numéroter en commentant la ligne \texttt{\\setcounter\{secnumdepth\}\{2\}} dans le fichier \texttt{preamble.tex}.
Vous remarquerez que les sous-sous-sections ne sont pas numérotées par défaut. C'est un choix de style\index{style}. Vous pouvez les numéroter en commentant la ligne \texttt{\\setcounter\{secnumdepth\}\{2\}} dans le fichier \texttt{preamble.tex}.
Vous remarquerez que les sous-sous-sections ne sont pas numérotées par défaut. C'est un choix de style\index{style}. Vous pouvez les numéroter en commentant la ligne \texttt{\\setcounter\{secnumdepth\}\{2\}} dans le fichier \texttt{preamble.tex}.
Vous remarquerez que les sous-sous-sections ne sont pas numérotées par défaut. C'est un choix de style\index{style}. Vous pouvez les numéroter en commentant la ligne \texttt{\\setcounter\{secnumdepth\}\{2\}} dans le fichier \texttt{preamble.tex}.
Vous remarquerez que les sous-sous-sections ne sont pas numérotées par défaut. C'est un choix de style\index{style}. Vous pouvez les numéroter en commentant la ligne \texttt{\\setcounter\{secnumdepth\}\{2\}} dans le fichier \texttt{preamble.tex}.
Vous remarquerez que les sous-sous-sections ne sont pas numérotées par défaut. C'est un choix de style\index{style}. Vous pouvez les numéroter en commentant la ligne \texttt{\\setcounter\{secnumdepth\}\{2\}} dans le fichier \texttt{preamble.tex}.
Vous remarquerez que les sous-sous-sections ne sont pas numérotées par défaut. C'est un choix de style\index{style}. Vous pouvez les numéroter en commentant la ligne \texttt{\\setcounter\{secnumdepth\}\{2\}} dans le fichier \texttt{preamble.tex}.
Vous remarquerez que les sous-sous-sections ne sont pas numérotées par défaut. C'est un choix de style\index{style}. Vous pouvez les numéroter en commentant la ligne \texttt{\\setcounter\{secnumdepth\}\{2\}} dans le fichier \texttt{preamble.tex}.
Vous remarquerez que les sous-sous-sections ne sont pas numérotées par défaut. C'est un choix de style\index{style}. Vous pouvez les numéroter en commentant la ligne \texttt{\\setcounter\{secnumdepth\}\{2\}} dans le fichier \texttt{preamble.tex}.
Vous remarquerez que les sous-sous-sections ne sont pas numérotées par défaut. C'est un choix de style\index{style}. Vous pouvez les numéroter en commentant la ligne \texttt{\\setcounter\{secnumdepth\}\{2\}} dans le fichier \texttt{preamble.tex}.
Vous remarquerez que les sous-sous-sections ne sont pas numérotées par défaut. C'est un choix de style\index{style}. Vous pouvez les numéroter en commentant la ligne \texttt{\\setcounter\{secnumdepth\}\{2\}} dans le fichier \texttt{preamble.tex}.
Vous remarquerez que les sous-sous-sections ne sont pas numérotées par défaut. C'est un choix de style\index{style}. Vous pouvez les numéroter en commentant la ligne \texttt{\\setcounter\{secnumdepth\}\{2\}} dans le fichier \texttt{preamble.tex}.
Vous remarquerez que les sous-sous-sections ne sont pas numérotées par défaut. C'est un choix de style\index{style}. Vous pouvez les numéroter en commentant la ligne \texttt{\\setcounter\{secnumdepth\}\{2\}} dans le fichier \texttt{preamble.tex}.
Vous remarquerez que les sous-sous-sections ne sont pas numérotées par défaut. C'est un choix de style\index{style}. Vous pouvez les numéroter en commentant la ligne \texttt{\\setcounter\{secnumdepth\}\{2\}} dans le fichier \texttt{preamble.tex}.
Vous remarquerez que les sous-sous-sections ne sont pas numérotées par défaut. C'est un choix de style\index{style}. Vous pouvez les numéroter en commentant la ligne \texttt{\\setcounter\{secnumdepth\}\{2\}} dans le fichier \texttt{preamble.tex}.
Vous remarquerez que les sous-sous-sections ne sont pas numérotées par défaut. C'est un choix de style\index{style}. Vous pouvez les numéroter en commentant la ligne \texttt{\\setcounter\{secnumdepth\}\{2\}} dans le fichier \texttt{preamble.tex}.
Vous remarquerez que les sous-sous-sections ne sont pas numérotées par défaut. C'est un choix de style\index{style}. Vous pouvez les numéroter en commentant la ligne \texttt{\\setcounter\{secnumdepth\}\{2\}} dans le fichier \texttt{preamble.tex}.
Vous remarquerez que les sous-sous-sections ne sont pas numérotées par défaut. C'est un choix de style\index{style}. Vous pouvez les numéroter en commentant la ligne \texttt{\\setcounter\{secnumdepth\}\{2\}} dans le fichier \texttt{preamble.tex}.
Vous remarquerez que les sous-sous-sections ne sont pas numérotées par défaut. C'est un choix de style\index{style}. Vous pouvez les numéroter en commentant la ligne \texttt{\\setcounter\{secnumdepth\}\{2\}} dans le fichier \texttt{preamble.tex}.
